\documentclass[a4paper,10pt,twoside,openright]{report}
%%%%%%%%%%%%%%%%%%%%%%%%%%%%%%%%%%%%%%%%%%%%%%%%
% Language, Encoding and Fonts
% http://en.wikibooks.org/wiki/LaTeX/Internationalization
%%%%%%%%%%%%%%%%%%%%%%%%%%%%%%%%%%%%%%%%%%%%%%%%
% Select encoding of your inputs. Depends on
% your operating system and its default input
% encoding. Typically, you should use
%   Linux  : utf8 (most modern Linux distributions)
%            latin1 
%   Windows: ansinew
%            latin1 (works in most cases)
%   Mac    : applemac
% Notice that you can manually change the input
% encoding of your files by selecting "save as"
% an select the desired input encoding. 
\usepackage[utf8]{inputenc}
% Make latex understand and use the typographic
% rules of the language used in the document.
\usepackage[danish,english]{babel}
% Use the vector font Latin Modern which is going
% to be the default font in latex in the future.
\usepackage{lmodern}
% Choose the font encoding
\usepackage[T1]{fontenc}


%%%%%%%%%%%%%%%%%%%%%%%%%%%%%%%%%%%%%%%%%%%%%%%%
% Graphics and Tables
% http://en.wikibooks.org/wiki/LaTeX/Importing_Graphics
% http://en.wikibooks.org/wiki/LaTeX/Tables
%%%%%%%%%%%%%%%%%%%%%%%%%%%%%%%%%%%%%%%%%%%%%%%%
% The standard graphics inclusion package
\usepackage{graphicx}
\DeclareGraphicsExtensions{% set the priority
    .pdf,.PDF,%
    .png,.PNG,%
    .jpg,.mps,.jpeg,.jbig2,.jb2,.JPG,.JPEG,.JBIG2,.JB2}
% Create graphics using PGF/TikZ
\usepackage{tikz}
\usetikzlibrary{matrix,decorations.pathreplacing,calc,arrows}
\usetikzlibrary{external}
\tikzexternalize[
  mode=convert with system call,% or graphics if exists, no graphics, only graphics
  figure list=true]
% Set up how figure and table captions are displayed
\usepackage{caption}
\captionsetup{%
  font=footnotesize,% set font size to footnotesize
  labelfont=bf % bold label (e.g., Figure 3.2) font
}
% Make the standard latex tables look so much better
\usepackage{cellspace,array,booktabs}
% Add vertical space in the tables
\addtolength\cellspacetoplimit{2pt}
\addtolength\cellspacebottomlimit{2pt}
% Enable sideway tables and figures
\usepackage{rotating}
% Color package
\usepackage{xcolor}
\definecolor{aaublue}{rgb}{0.00,0.41,0.66}% dark blue
\ifx\type\undefined % screen version - this is the default
  \colorlet{titlecolor}{aaublue}
  \colorlet{linkcolor}{aaublue}
  \colorlet{bibcolor}{aaublue}
  \colorlet{urlcolor}{aaublue}
  \colorlet{examplecolor}{aaublue!50}
\else % print version
  \colorlet{titlecolor}{black}
  \colorlet{linkcolor}{black}
  \colorlet{bibcolor}{black}
  \colorlet{urlcolor}{black}
  \colorlet{examplecolor}{gray}
\fi
% Framed and shaded environments
\usepackage{framed}

%%%%%%%%%%%%%%%%%%%%%%%%%%%%%%%%%%%%%%%%%%%%%%%%
% Page Layout
% http://en.wikibooks.org/wiki/LaTeX/Page_Layout
%%%%%%%%%%%%%%%%%%%%%%%%%%%%%%%%%%%%%%%%%%%%%%%%
% Change margins, papersize, etc of the document
\usepackage{geometry}
% Modify how \chapter, \section, etc. look
% The titlesec package is very configureable
\usepackage{titlesec}
\titleformat*{\section}{\normalfont\Large\bfseries\color{titlecolor}}
\titleformat*{\subsection}{\normalfont\large\bfseries\color{titlecolor}}
\titleformat*{\subsubsection}{\normalfont\normalsize\bfseries\color{titlecolor}}
%\titleformat*{\paragraph}{\normalfont\normalsize\bfseries\color{titlecolor}}
%\titleformat*{\subparagraph}{\normalfont\normalsize\bfseries\color{titlecolor}}
% Change the headers and footers
\usepackage{fancyhdr}
\pagestyle{fancy}
\fancyhf{} %delete everything
\fancyhead[RE]{\small\nouppercase\leftmark} %even page - chapter title
\fancyhead[LO]{\small\nouppercase\rightmark} %uneven page - section title
\fancyhead[LE,RO]{\thepage} %page number on all pages
% Do not stretch the content of a page. Instead,
% insert white space at the bottom of the page
\raggedbottom
% Enable arithmetics with length. Useful when
% typesetting the layout.
\usepackage{calc}

%%%%%%%%%%%%%%%%%%%%%%%%%%%%%%%%%%%%%%%%%%%%%%%%
% Mathematics
% http://en.wikibooks.org/wiki/LaTeX/Mathematics
%%%%%%%%%%%%%%%%%%%%%%%%%%%%%%%%%%%%%%%%%%%%%%%%
% Defines new environments such as equation,
% align and split 
\usepackage{amsmath}
% Adds new math symbols
\usepackage{amssymb}
% Enables e.g. \iff between lines
\usepackage{mathtools}
% When using thmmarks, amsmath must be an option as well. Otherwise \eqref doesn't work anymore.
\usepackage[framed,amsmath,thmmarks]{ntheorem}

%%%%%%%%%%%%%%%%%%%%%%%%%%%%%%%%%%%%%%%%%%%%%%%%
% Bibliography
% http://en.wikibooks.org/wiki/LaTeX/Bibliography_Management
%%%%%%%%%%%%%%%%%%%%%%%%%%%%%%%%%%%%%%%%%%%%%%%%
% Appearance of the bibliography
\bibliographystyle{IEEEtran}

%%%%%%%%%%%%%%%%%%%%%%%%%%%%%%%%%%%%%%%%%%%%%%%%
% Misc
%%%%%%%%%%%%%%%%%%%%%%%%%%%%%%%%%%%%%%%%%%%%%%%%
% Add bibliography and index to the table of
% contents
\usepackage[nottoc]{tocbibind}
% Add the command \pageref{LastPage} which refers to the
% page number of the last page
\usepackage{lastpage}
% Change the chapter name
\addto\captionsenglish{%this line is required when using the babel package
  \renewcommand\chaptername{Lecture}
}
% Todo notes
% http://www.tex.ac.uk/tex-archive/macros/latex/contrib/todonotes/todonotes.pdf
\usepackage[colorinlistoftodos,shadow,disable]{todonotes}


%%%%%%%%%%%%%%%%%%%%%%%%%%%%%%%%%%%%%%%%%%%%%%%%
% Hyperlinks
% http://en.wikibooks.org/wiki/LaTeX/Hyperlinks
%%%%%%%%%%%%%%%%%%%%%%%%%%%%%%%%%%%%%%%%%%%%%%%%
% Enable hyperlinks and insert info into the pdf
% file. Hypperref should be loaded as one of the 
% last packages

\usepackage{hyperref}
\hypersetup{%
	pdfpagelabels=true,%
	plainpages=false,%
	pdfauthor={Jesper Kjær Nielsen and Søren Holdt Jensen},%
	pdftitle={Lecture Notes in Adaptive Filters},%
	pdfsubject={Adaptive Filtering},%
	bookmarksnumbered=true,%
	colorlinks,%
	citecolor=bibcolor,%
	filecolor=urlcolor,%
	linkcolor=linkcolor,%
	urlcolor=urlcolor,%
	pdfstartview=FitH%
}
