\chapter*{Preface\markboth{Preface}{Preface}}\label{ch:preface}
\addcontentsline{toc}{chapter}{Preface}
{
  \definecolor{shadecolor}{gray}{.87}
  \tt
  \begin{shaded}
\begin{verbatim}
\*
 * Your warranty is now void.
 *
 * By installing this knowledge onto your brain, you accept that we, the 
 * authors of the present lecture notes, are not responsible for potential
 * confusion, wasted time, failed exams, or misguided missiles.
 *
 * You will most likely encounter unclear statements and remarks, poor English
 * usage, and errors when you read the present notes. Therefore, use these
 * notes at your own risk, and please help us correct these problems by giving
 * us some feedback.
 *\
\end{verbatim}
  \end{shaded}
}

\noindent The present lecture notes were written for the annual course on adaptive filters at Aalborg University. The content of the course is now a part of the annual course called \textit{Array and Sensor Signal Processing}. The notes are written for the lecturer, but they may also be useful to the student as a supplement to his/her favourite textbook. Consequently, the notes are very concise and contain only what we believe to be the basics of adaptive filtering. Moreover, we have also made some important simplifications.
\begin{enumerate}
  \item We use real-valued numbers and not complex-valued numbers. Although the latter is more general, it is less confusing and leads to fewer errors when real-valued numbers are used.
  \item We only consider FIR adaptive filters.
  \item The signals have zero mean. This is a standard assumption used in most textbooks.
\end{enumerate}
Each of the lectures contains an amount of material suited for a lecture lasting for approximately 90 minutes. The appendices contain a summary and some supplementary material.

These lecture notes are always work in progress. Therefore, if you have found an error, have a suggestion for a better statement and/or explanation, or just want to give us some feedback, then do not hesitate to contact us.
