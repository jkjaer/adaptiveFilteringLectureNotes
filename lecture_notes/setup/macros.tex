%various names
\def\TikZ{Ti{\em k}Z\@}
\def\PGF{{\scriptsize PGF}\@}
\def\PGFPLOTS{{\scriptsize PGFPLOTS}\@}
\def\GNUPLOT{{\scriptsize GNUPLOT}\@}
%shorter math commands
\newcommand{\bmath}{\begin{equation}}
\newcommand{\emath}{\end{equation}}
\newcommand{\bmathnt}{\begin{equation*}}%no tags
\newcommand{\emathnt}{\end{equation*}}%no tags
\newcommand{\bbmtx}{\begin{bmatrix}}
\newcommand{\ebmtx}{\end{bmatrix}}
\newcommand{\vect}[1]{\boldsymbol{#1}}
\newcommand{\eqpage}[1]{Eq.~\ref{#1} page~\pageref{#1}}
\newcommand{\eq}[1]{Eq.~\eqref{#1}}
% \argmax and \argmin math operator (http://en.wikipedia.org/wiki/Arg_max)
\DeclareMathOperator*{\argmax}{arg\,max}
\DeclareMathOperator*{\argmin}{arg\,min}
%framed aligned equations
\newlength\dlf
\newcommand\alignedbox[2]{
  % #1 = before alignment
  % #2 = after alignment
  &
  \begingroup
  \settowidth\dlf{$\displaystyle #1$}
  \addtolength\dlf{\fboxsep+\fboxrule}
  \hspace{-\dlf}
  \boxed{#1 #2}
  \endgroup
}
\newcommand{\tr}[1]{\textup{tr}(#1)} % trace of a matrix
\newcommand{\rank}[1]{\textup{rank}(#1)} % rank of a matrix
\newcommand{\diag}[1]{\textup{diag}(#1)} % diagonal matrix



%%%tikz
\newcommand{\inputTikZ}[1]{%
  \tikzsetnextfilename{#1}
  \input{#1.tikz}%
}

%styles for block diagrams
\tikzstyle{combnode} = [draw,fill=gray!20,circle]
\tikzstyle{prodnode} = [draw,fill=gray!20,isosceles triangle]
\tikzstyle{dotnode} = [circle,inner sep=0pt,minimum size=3pt,fill=black]
\tikzstyle{blocknode} = [draw,fill=gray!20,minimum height=1.5cm,minimum width=3cm,text width=2.5cm,text centered]
%\newcommand{\prodnode}[2]{%
%  \node (#1) [draw,fill=gray!20,isosceles triangle] {#2};
%}
%\newcommand{\dotnode}[1]{%
%  \node (#1) [circle,inner sep=0pt,minimum size=3pt,fill=black] {};
%}
%\newcommand{\blocknode}[2]{%
%  \node (#1) [draw,fill=gray!20,minimum height=1.5cm,text width=3cm,text centered] {#2};
%}

%%%%%%%%%%%%%%%%%%%%%%%%%%%%%%%%%%%%%%%%%%%%%%%%
% An example environment
%%%%%%%%%%%%%%%%%%%%%%%%%%%%%%%%%%%%%%%%%%%%%%%%
\theoremheaderfont{\normalfont\bfseries}
\theorembodyfont{\normalfont}
\theoremstyle{break}
\def\theoremframecommand{{\color{examplecolor}\vrule width 5pt \hspace{5pt}}}
\newshadedtheorem{exa}{Example}[chapter]
\newenvironment{example}[1]{%
		\begin{exa}[#1]
}{%
		\end{exa}
}
